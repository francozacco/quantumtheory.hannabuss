\documentclass[11pt]{article}
\usepackage{amssymb}
\usepackage{amsthm}
\usepackage{enumitem}
\usepackage{amsmath}
\usepackage{bm}
\usepackage{adjustbox}
\usepackage{mathrsfs}
\usepackage{graphicx}
\usepackage{siunitx}
\usepackage[mathscr]{euscript}

\title{\textbf{Solved selected problems of An Introduction to Quantum Theory
by Keith Hannabuss}}
\author{Franco Zacco}
\date{}

\addtolength{\topmargin}{-3cm}
\addtolength{\textheight}{3cm}

\newcommand{\hatr}{\bm{\hat{r}}}
\newcommand{\hatx}{\bm{\hat{x}}}
\newcommand{\haty}{\bm{\hat{y}}}
\newcommand{\hatz}{\bm{\hat{z}}}
\newcommand{\hatth}{\bm{\hat{\theta}}}
\newcommand{\hatphi}{\bm{\hat{\phi}}}
\newcommand{\hatrho}{\bm{\hat{\rho}}}
\theoremstyle{definition}
\newtheorem*{solution*}{Solution}
\renewcommand*{\proofname}{Solution}

\begin{document}
\maketitle
\thispagestyle{empty}

\section*{Chapter 1 - Introduction}
\begin{proof}{1.1}
\begin{itemize}
\item [(i)] A frequency of $200 kHz$ corresponds to an energy of
\begin{align*}
    E = \hbar \omega = 1.0546 \times 10^{-34} \cdot 2\pi \cdot 2 \times 10^5
    = 1.3252 \times 10^{-28}~J
\end{align*}
\item [(ii)] A frequency of $4.95 \times 10^{14} Hz$ corresponds to an energy
of
\begin{align*}
    E = \hbar \omega = 1.0546 \times 10^{-34} \cdot 2\pi \cdot 4.95 \times 10^{14}
    = 3.2799 \times 10^{-19}~J
\end{align*}
\item [(iii)] A frequency of $10^{20} Hz$ corresponds to an energy of
\begin{align*}
    E = \hbar \omega = 1.0546 \times 10^{-34} \cdot 2\pi \cdot 10^{20}
    = 6.62624\times 10^{-14}~J
\end{align*}
\item [(iv)] A frequency of $10^{23} Hz$ corresponds to an energy of
\begin{align*}
    E = \hbar \omega = 1.0546 \times 10^{-34} \cdot 2\pi \cdot 10^{23}
    = 6.62624\times 10^{-11}~J
\end{align*}
\end{itemize}
\end{proof}

\cleardoublepage
\begin{proof}{1.2}
Let a radio station broadcast on a frequency of $200~kHz$ then each photon
generated has an energy of
\begin{align*}
    E = 1.0546 \times 10^{-34} \cdot 2\pi \cdot 2 \times 10^5
    = 1.3252 \times 10^{-28}~J
\end{align*}
So a $200~kW = 200~kJ/s$ transmitter generates 
\begin{align*}
    \frac{200\times 10^3}{1.3252 \times 10^{-28}}
    = 1.5092 \times 10^{33}
\end{align*}
photons per second.

Let us assume the radio station broadcasts photons in every direction then 
the radio station broadcasts the following number of photons per meter squared
per second
\begin{align*}
    \frac{1.5092 \times 10^{33}}{4\pi\cdot (1\times 10^6~m)^2}
    = 1.2009 \times 10^{20}
\end{align*}

Finally if the aerial were on a space probe at a distance of 3000 millon km of
the earth the number of photons per meter squared per second will be
\begin{align*}
    \frac{1.5092 \times 10^{33}}{4\pi\cdot (3 \times 10^{12}~m)^2}
    = 1.3344 \times 10^{7}
\end{align*}
\end{proof}
\end{document}